\documentclass[12pt]{article}
\newcommand{\blank}[1]{\hspace*{#1}\linebreak[0]}
\usepackage{xcolor}  
\pagecolor[rgb]{0, 0.082, 0.180}  
\color[rgb]{0.921, 0.623, 0}  
\begin{document}
\begin{titlepage}
    \begin{center}
        \vspace*{1cm}
        \Huge
        \textbf{Plutarchus Documentation}
        \\
        \LARGE
        \vspace{0.5cm}
         Beta 0.1
         \\
        \vspace{0.5cm}
        \vspace{8.5cm}
        \vspace{0.8cm}
        \hspace{2cm}
        \blank{6cm} August 30, 2019
    \end{center}
 \end{titlepage}
 
    \tableofcontents
    \newpage
    \section{Introduction}

    Plutarchus is an open source personal website building tool. It takes a JSON file and generates a website using the data in the JSON file. The JSON file is the dataset of the user's work. The websites that Plutarchus generates are static HTMLs, however Javascript and CSS are integrated to make then at par with modern web designs. Plutarchus v0.1 is designed to build resume websites.
    The examples and test cases are websites of academics and hence same are the target audience for v 0.1.      
 
    \subsection{Why Plutarchus}

    There are a number of free website building tools available online. The key idea of the Plutarchus that separate it from other is that it generates a website from just a JSON file. Firstly, the user doesn't have to do any web design. Second, the user can maintain a JSON file of his work by simply adding entries to as new work is produced. The JSON file can be read as a dataset in various programming language for data analytics, etc.  
    

    \subsection{Examples}
    \section{Getting Started}
    \subsection{Writing a JSON File}
    \subsection{Examples}
    \subsubsection{Getting Help}
    \subsection{Website Generation}
    \subsubsection{Examples}
    \subsubsection{Command Line Interface}
    \subsubsection{Graphics User Interface}
    \subsection{Getting Help}
    \section{Themes}
    \subsection{Themes in V0.1}
    \subsection{Theme0}
\end{document}